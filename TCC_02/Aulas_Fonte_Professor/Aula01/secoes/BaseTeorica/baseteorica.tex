\section{Fundamentação Teórica}

\begin{frame}	
	\begin{block}{Fundamentação Teórica}
		 \begin{itemize}
			  \item Conhecimento necessário para compreender seu trabalho
			  \item Técnicas estatísticas
			  \item Algoritmos complexos
			  \item Equações
			  \item Provas de teoremas matemáticos
			  \item Frameworks para gestão de projetos
			  \item Exemplo de mestrado com fundamentação teórica \href{http://www.teses.usp.br/teses/disponiveis/100/100131/tde-19042016-140611/pt-br.php}{\color{blue}{Mestrado}} \cite{KHOURI_2016}
		  \end{itemize}
	\end{block}
	
Informação necessária, não trivial, para permitir a leitura do projeto e mostrar que você compreende a fundo o problema a ser resolvido e as soluções para ele.
\end{frame}
