\section{Boas Fontes de Informação}

\begin{frame}	
	\begin{block}{Boas Fontes de Informação}	
		\begin{itemize}
			\item Artigos científicos com revisão em pares (blind)
			\item Livros
			\item Teses e dissertações
		\end{itemize}
	\end{block}
\end{frame}

\begin{frame}	
	\begin{block}{Exemplos de boas fontes de informação}	
		\begin{itemize}
			\item \href{https://www.jstor.org}{\color{blue}{JStor}} 
			\item \href{https://ieeexplore.ieee.org/Xplore/home.jsp}{\color{blue}{IEEExplore}} 
			\item \href{https://dl.acm.org}{\color{blue}{ACMDL}} 
			\item \href{https://www.scopus.com/home.uri}{\color{blue}{Scopus}} 
			\item \href{https://link.springer.com}{\color{blue}{Springer}} 
			\item \href{https://www.sciencedirect.com}{\color{blue}{ScienceDirect}} 
			\item \href{http://www.scielo.br/?lng=pt}{\color{blue}{Scielo}} 
			\item \href{http://www.periodicos.capes.gov.br}{\color{blue}{CAPES}} 
			\item \href{http://www.bibliotecadigital.unicamp.br}{\color{blue}{Teses Unicamp}} 
			\item \href{http://www.teses.usp.br}{\color{blue}{Teses USP}} 
		\end{itemize}
	\end{block}
\end{frame}

\begin{frame}	
	\begin{block}{Fontes Questionáveis de Informação}	
		\begin{itemize}
			\item Wikipedia
			\item Google
			\item Google Scholar (Como assim professor?)
			\item Sites no geral 
		\end{itemize}
	\end{block}
\end{frame}

